% Options for packages loaded elsewhere
\PassOptionsToPackage{unicode}{hyperref}
\PassOptionsToPackage{hyphens}{url}
\PassOptionsToPackage{dvipsnames,svgnames,x11names}{xcolor}
%
\documentclass[
  letterpaper,
  DIV=11,
  numbers=noendperiod]{scrartcl}

\usepackage{amsmath,amssymb}
\usepackage{iftex}
\ifPDFTeX
  \usepackage[T1]{fontenc}
  \usepackage[utf8]{inputenc}
  \usepackage{textcomp} % provide euro and other symbols
\else % if luatex or xetex
  \usepackage{unicode-math}
  \defaultfontfeatures{Scale=MatchLowercase}
  \defaultfontfeatures[\rmfamily]{Ligatures=TeX,Scale=1}
\fi
\usepackage{lmodern}
\ifPDFTeX\else  
    % xetex/luatex font selection
\fi
% Use upquote if available, for straight quotes in verbatim environments
\IfFileExists{upquote.sty}{\usepackage{upquote}}{}
\IfFileExists{microtype.sty}{% use microtype if available
  \usepackage[]{microtype}
  \UseMicrotypeSet[protrusion]{basicmath} % disable protrusion for tt fonts
}{}
\makeatletter
\@ifundefined{KOMAClassName}{% if non-KOMA class
  \IfFileExists{parskip.sty}{%
    \usepackage{parskip}
  }{% else
    \setlength{\parindent}{0pt}
    \setlength{\parskip}{6pt plus 2pt minus 1pt}}
}{% if KOMA class
  \KOMAoptions{parskip=half}}
\makeatother
\usepackage{xcolor}
\setlength{\emergencystretch}{3em} % prevent overfull lines
\setcounter{secnumdepth}{-\maxdimen} % remove section numbering
% Make \paragraph and \subparagraph free-standing
\ifx\paragraph\undefined\else
  \let\oldparagraph\paragraph
  \renewcommand{\paragraph}[1]{\oldparagraph{#1}\mbox{}}
\fi
\ifx\subparagraph\undefined\else
  \let\oldsubparagraph\subparagraph
  \renewcommand{\subparagraph}[1]{\oldsubparagraph{#1}\mbox{}}
\fi

\usepackage{color}
\usepackage{fancyvrb}
\newcommand{\VerbBar}{|}
\newcommand{\VERB}{\Verb[commandchars=\\\{\}]}
\DefineVerbatimEnvironment{Highlighting}{Verbatim}{commandchars=\\\{\}}
% Add ',fontsize=\small' for more characters per line
\usepackage{framed}
\definecolor{shadecolor}{RGB}{241,243,245}
\newenvironment{Shaded}{\begin{snugshade}}{\end{snugshade}}
\newcommand{\AlertTok}[1]{\textcolor[rgb]{0.68,0.00,0.00}{#1}}
\newcommand{\AnnotationTok}[1]{\textcolor[rgb]{0.37,0.37,0.37}{#1}}
\newcommand{\AttributeTok}[1]{\textcolor[rgb]{0.40,0.45,0.13}{#1}}
\newcommand{\BaseNTok}[1]{\textcolor[rgb]{0.68,0.00,0.00}{#1}}
\newcommand{\BuiltInTok}[1]{\textcolor[rgb]{0.00,0.23,0.31}{#1}}
\newcommand{\CharTok}[1]{\textcolor[rgb]{0.13,0.47,0.30}{#1}}
\newcommand{\CommentTok}[1]{\textcolor[rgb]{0.37,0.37,0.37}{#1}}
\newcommand{\CommentVarTok}[1]{\textcolor[rgb]{0.37,0.37,0.37}{\textit{#1}}}
\newcommand{\ConstantTok}[1]{\textcolor[rgb]{0.56,0.35,0.01}{#1}}
\newcommand{\ControlFlowTok}[1]{\textcolor[rgb]{0.00,0.23,0.31}{#1}}
\newcommand{\DataTypeTok}[1]{\textcolor[rgb]{0.68,0.00,0.00}{#1}}
\newcommand{\DecValTok}[1]{\textcolor[rgb]{0.68,0.00,0.00}{#1}}
\newcommand{\DocumentationTok}[1]{\textcolor[rgb]{0.37,0.37,0.37}{\textit{#1}}}
\newcommand{\ErrorTok}[1]{\textcolor[rgb]{0.68,0.00,0.00}{#1}}
\newcommand{\ExtensionTok}[1]{\textcolor[rgb]{0.00,0.23,0.31}{#1}}
\newcommand{\FloatTok}[1]{\textcolor[rgb]{0.68,0.00,0.00}{#1}}
\newcommand{\FunctionTok}[1]{\textcolor[rgb]{0.28,0.35,0.67}{#1}}
\newcommand{\ImportTok}[1]{\textcolor[rgb]{0.00,0.46,0.62}{#1}}
\newcommand{\InformationTok}[1]{\textcolor[rgb]{0.37,0.37,0.37}{#1}}
\newcommand{\KeywordTok}[1]{\textcolor[rgb]{0.00,0.23,0.31}{#1}}
\newcommand{\NormalTok}[1]{\textcolor[rgb]{0.00,0.23,0.31}{#1}}
\newcommand{\OperatorTok}[1]{\textcolor[rgb]{0.37,0.37,0.37}{#1}}
\newcommand{\OtherTok}[1]{\textcolor[rgb]{0.00,0.23,0.31}{#1}}
\newcommand{\PreprocessorTok}[1]{\textcolor[rgb]{0.68,0.00,0.00}{#1}}
\newcommand{\RegionMarkerTok}[1]{\textcolor[rgb]{0.00,0.23,0.31}{#1}}
\newcommand{\SpecialCharTok}[1]{\textcolor[rgb]{0.37,0.37,0.37}{#1}}
\newcommand{\SpecialStringTok}[1]{\textcolor[rgb]{0.13,0.47,0.30}{#1}}
\newcommand{\StringTok}[1]{\textcolor[rgb]{0.13,0.47,0.30}{#1}}
\newcommand{\VariableTok}[1]{\textcolor[rgb]{0.07,0.07,0.07}{#1}}
\newcommand{\VerbatimStringTok}[1]{\textcolor[rgb]{0.13,0.47,0.30}{#1}}
\newcommand{\WarningTok}[1]{\textcolor[rgb]{0.37,0.37,0.37}{\textit{#1}}}

\providecommand{\tightlist}{%
  \setlength{\itemsep}{0pt}\setlength{\parskip}{0pt}}\usepackage{longtable,booktabs,array}
\usepackage{calc} % for calculating minipage widths
% Correct order of tables after \paragraph or \subparagraph
\usepackage{etoolbox}
\makeatletter
\patchcmd\longtable{\par}{\if@noskipsec\mbox{}\fi\par}{}{}
\makeatother
% Allow footnotes in longtable head/foot
\IfFileExists{footnotehyper.sty}{\usepackage{footnotehyper}}{\usepackage{footnote}}
\makesavenoteenv{longtable}
\usepackage{graphicx}
\makeatletter
\def\maxwidth{\ifdim\Gin@nat@width>\linewidth\linewidth\else\Gin@nat@width\fi}
\def\maxheight{\ifdim\Gin@nat@height>\textheight\textheight\else\Gin@nat@height\fi}
\makeatother
% Scale images if necessary, so that they will not overflow the page
% margins by default, and it is still possible to overwrite the defaults
% using explicit options in \includegraphics[width, height, ...]{}
\setkeys{Gin}{width=\maxwidth,height=\maxheight,keepaspectratio}
% Set default figure placement to htbp
\makeatletter
\def\fps@figure{htbp}
\makeatother

\KOMAoption{captions}{tableheading}
\makeatletter
\@ifpackageloaded{caption}{}{\usepackage{caption}}
\AtBeginDocument{%
\ifdefined\contentsname
  \renewcommand*\contentsname{Table of contents}
\else
  \newcommand\contentsname{Table of contents}
\fi
\ifdefined\listfigurename
  \renewcommand*\listfigurename{List of Figures}
\else
  \newcommand\listfigurename{List of Figures}
\fi
\ifdefined\listtablename
  \renewcommand*\listtablename{List of Tables}
\else
  \newcommand\listtablename{List of Tables}
\fi
\ifdefined\figurename
  \renewcommand*\figurename{Figure}
\else
  \newcommand\figurename{Figure}
\fi
\ifdefined\tablename
  \renewcommand*\tablename{Table}
\else
  \newcommand\tablename{Table}
\fi
}
\@ifpackageloaded{float}{}{\usepackage{float}}
\floatstyle{ruled}
\@ifundefined{c@chapter}{\newfloat{codelisting}{h}{lop}}{\newfloat{codelisting}{h}{lop}[chapter]}
\floatname{codelisting}{Listing}
\newcommand*\listoflistings{\listof{codelisting}{List of Listings}}
\makeatother
\makeatletter
\makeatother
\makeatletter
\@ifpackageloaded{caption}{}{\usepackage{caption}}
\@ifpackageloaded{subcaption}{}{\usepackage{subcaption}}
\makeatother
\ifLuaTeX
  \usepackage{selnolig}  % disable illegal ligatures
\fi
\usepackage{bookmark}

\IfFileExists{xurl.sty}{\usepackage{xurl}}{} % add URL line breaks if available
\urlstyle{same} % disable monospaced font for URLs
\hypersetup{
  pdftitle={African American Periodical Poetry (1900-1928)},
  pdfauthor={Amardeep Singh and Kate Hennessey},
  colorlinks=true,
  linkcolor={blue},
  filecolor={Maroon},
  citecolor={Blue},
  urlcolor={Blue},
  pdfcreator={LaTeX via pandoc}}

\title{African American Periodical Poetry (1900-1928)}
\author{Amardeep Singh and Kate Hennessey}
\date{2024-05-17}

\begin{document}
\maketitle

\renewcommand*\contentsname{Table of contents}
{
\hypersetup{linkcolor=}
\setcounter{tocdepth}{5}
\tableofcontents
}
\section{Data Essay}

\subsection{Introduction}\label{introduction}

This dataset is a collection of poetry published in magazines by African
American writers in the early twentieth century. The collection
currently contains about 750 poems, all of which are out of copyright in
the U.S., with the bulk of the poetry published in fourteen different
magazines, including both Black-oriented and edited magazines like
\emph{The Crisis}, \emph{Opportunity}, and \emph{Black Opals} as well as
`mainstream' (predominantly White) magazines like \emph{Survey Graphic},
\emph{The Liberator}, and \emph{Palms}. The dataset includes information
on authors' biographies, poetic form, topics and themes, the names of
editors (including guest editors and co-editors), as well as
republication information.

\begin{Shaded}
\begin{Highlighting}[]
\NormalTok{//| echo: false}
\NormalTok{Inputs.table(search, \{}
\NormalTok{   width: [50, 50, 50, 50, 50],}
\NormalTok{  layout: "fixed",}
\NormalTok{  rows: 25,}
\NormalTok{  sort: "year",}
\NormalTok{  reverse: false,}

\NormalTok{  format: \{}
\NormalTok{    /*RecreationVisits: x =\textgreater{} d3.format(\textquotesingle{}.2s\textquotesingle{})(x),*/}
\NormalTok{    year: x =\textgreater{} d3.timeFormat(x),}
\NormalTok{    text: x =\textgreater{} \{}
\NormalTok{      const words = x.split(\textquotesingle{} \textquotesingle{});}
\NormalTok{      return words.slice(0, 15).join(\textquotesingle{} \textquotesingle{}) + (words.length \textgreater{} 15 ? \textquotesingle{}...\textquotesingle{} : \textquotesingle{}\textquotesingle{});}
\NormalTok{    \}}
\NormalTok{    // author\_birth: x =\textgreater{} d3.timeFormat(x),}
\NormalTok{    // author\_death: x =\textgreater{} d3.timeFormat(x),}
\NormalTok{    // gr\_num\_ratings\_rank: x =\textgreater{} html\textasciigrave{}\textless{}div style=\textquotesingle{}background:$\{color(x)\}\textquotesingle{}\textgreater{}$\{d3.format(\textquotesingle{}.2s\textquotesingle{})(x)\}\textless{}/div\textgreater{}\textasciigrave{}}
\NormalTok{  \}}
\NormalTok{\})}
\end{Highlighting}
\end{Shaded}

Magazines were a crucial part of the landscape of literary publishing in
the U.S. during the early twentieth century. Magazines with often very
large national readerships helped writers find their audience, and
opportunities to work as editors help position established writers in
positions of power and authority in the American publishing industry.

The broader goal is to collect materials from both predominantly Black
and predominantly White (or ``mainstream'') magazines, to enable
researchers to understand the changing shape of African American poetry
in and out of the American mainstream. We'll explain some specific
dimensions of the collection we've put together below, including how to
search and sort through the material. Under a separate tab, we'll also
suggest some possible classroom exercises that might make this
collection useful for students who are learning about African American
poetry for the first time. However, first we'll briefly introduce the
concept of African American American Periodical Poetry.

\subsection{African American Periodical Poetry: a Bit of
Background}\label{african-american-periodical-poetry-a-bit-of-background}

In their poetry, African American writers argued for Civil Rights,
challenged the unjust practices and racialized violence that were
prevalent in American life, and worked out new and distinctive aesthetic
concepts. At the beginning of the twentieth century, African American
poetry was not so much a `literary' movement as it was a part of popular
culture, with hugely popular and influential figures like Paul Laurence
Dunbar and an important element of African American Vernacular writing
and performance linked to the oral tradition (Ramey, \emph{A History of
African American Poetry}, 2019). However, by the 1920s, changes in the
style and function of the poetry was evident, and these changes can be
readily traced in poetry published in periodicals.

The overall conception of this project borrows from Modernist Periodical
Studies, led by scholars such as Sean Latham (``The Rise of Periodical
Studies,'' 2006), Robert Scholes and Clifford Wulfman (\emph{Modernism
in the Magazines}, 2010), and Adam McKible and Suzanne Churchill
(\emph{Little Magazines and Modernism}, 2005). It is also deeply
invested in the field that has come to be known as Black Digital
Humanities scholarship; we hope our work will be legible as in the vein
of Kim Gallon's ``technology of recovery'' (``Making a Case for the
Black Digital Humanities, 2016), where an important goal is to make
under-studied materials accessible and legible to a broad readership,
including readers outside of academia. Other scholarship that has
influenced our own includes the \emph{Colored Conventions Project}
(Foreman, Casey, and Patterson, 2021), and Roopika Risam's \emph{New
Digital Worlds}(2018). And more generally, we see our work as continuing
to fill in a continuing problem of the ``Archive Gap,'' where
minoritized writers often continue to be under-studied and under-curated
in major digital collections, with knock-on effects for quantitative
analysis based on textual corpora (see Singh, ``Beyond the Archive
Gap,'' 2020).

\subsection{Why These Particular Periodicals? The Black Press and the
Advent of ``Crossover''
Publications}\label{why-these-particular-periodicals-the-black-press-and-the-advent-of-crossover-publications}

We have attempted to build a comprehensive collection of African
American periodical poetry from the early 20th century, arranging it so
that visitors to this site can search and sort based on their own
research questions. We believe that the creation of this collection will
allow both serous scholars as well as students and those new to the
field to see the full richness and range of African American poetry from
this period in new ways.

\textbf{The Black Press.} First and foremost, we believe that in order
to understand how African American poetry developed and grew in the
years covered by this project, it's important to engage with the primary
venue through which that poetry was published and read -- the Black
press. For many poets in this Anthology, African American-oriented
periodicals were the first destination for poetry. Magazine publication
was how Black poets found readers and developed a sense of community,
and the collections we have produced for this project hopefully show
some powerful emerging networks and conversations about the role of
literature in civil rights activism, questions of aesthetics, and
questions of language. Also, at least until the mid-1920s, publication
in mainstream venues like \emph{Poetry} or \emph{Vanity Fair} was
extremely rare for African American poets, so venues like \emph{The
Crisis} and \emph{Opportunity} were often the best way to get published.

To be sure, the poetry published by these magazines is of quite varying
quality and importance to literary historians. Rather than evaluate or
curate the materials based on our particular tastes or preexisting
consensus, our aspiration here is to make \_all\_of the periodical
poetry we can find accessible to readers with the hope that it will be
helpful to different kinds of research projects and questions. Putting
everything together in one place, as we have attempted to do here, can
sometimes reveal alignments and patterns that might not have previously
been visible. What might readers discover? What new questions can we
ask?

\textbf{1900-1920: A somewhat overlooked era in African American
poetry.} The field of African American periodical poetry was quite
active in the first two decades of the 20th century, though many of the
magazines where poetry was published were not well-preserved, and are
today only available in fragments. Two important magazines from the
first decade of the twentieth century that have been been well-preserved
and digitized are \emph{Voice of the Negro} (Atlanta and Chicago,
1904-1907) and \emph{Colored American Magazine} (Boston and New York,
1900-1908). Both printed a considerable amount of poetry, and we have
included a substantial number of poems from these magazines in our
current collection (note: our collection of poetry from these two
magazines is not quite complete). W.E.B. Du Bois also edited two
magazines that appeared only briefly in this period, \emph{Moon
Illustrated Weekly} (1905-1906), and \emph{Horizon: A Journal of the
Color Line} (1907-1910). \emph{Horizon} did not appear to print much in
the way of poetry; \emph{Moon Illustrated Weekly} printed poems only
occasionally. Both magazines have only been partially preserved. At
present, those magazines are sparsely represented in our collection.

\emph{The Crisis} became a major vehicle for poetry publication starting
in 1910, under Du Bois' editorship. It was an especially important venue
for Georgia Douglas Johnson, who published more than 30 poems in the
magazine during this period; we also see substantial numbers of poems
published by writers like James D. Corrothers, Joseph S. Cotter, and
Lucian B. Watkins -- writers who were well known at this time, but who
subsequently dropped off the critical radar. Du Bois himself published
several notable poems in \emph{The Crisis} during this period, and these
poems are included in our collection. One particularly active theme in
late 1910s poetry is World War I (see the Glossary in a separate tab for
other themes).

\textbf{The 1920s: \emph{The Crisis}, \emph{Opportunity}, and \emph{The
Messenger}. } There were three very large and successful African
American magazines in the 1920s, all published in New York but with
national circulations: \emph{The Crisis}, \emph{Opportunity}, \_and
\emph{The Messenger}. With \emph{The Crisis}, there appears to have been
a steady build-up of interest in poetry through the 1910s, leading up to
Jessie Fauset's taking on the role of literary editor between 1919-1926.
With her work at \emph{The Crisis}, Fauset brought a serious literary
sensibility to the magazine and a considerable increase in quality.

With regards to \emph{Opportunity}: as biographers for writers like Zora
Neale Hurston and Langston Hughes have documented, the
racially-integrated \emph{Opportunity }``prize'' dinners in 1924, 1925,
and 1926 seemed to function at least in part as coming out parties for
emerging writers, helping to create a sense of excitement about African
American poetry both within the Black community and beyond it. White
editors and publishers like Paul Kellogg and Horace Liveright attended a
particularly important dinner at the Harlem Civic Club in March of 1924,
which led to the idea of the special issue devoted to African American
writing in \emph{Survey Graphic }(March 1925), guest-edited by Alain
Locke.

\textbf{The mid-1920s as tipping point for African American poetry.} As
we've developed our collection of African American periodical poetry,
it's become clearer and clearer that the mid- and late-1920s was a
special moment of growth and recognition. First and foremost, there is
the remarkable proliferation of African American general interest
magazines like \emph{The Crisis} and \emph{Opportunity}, as mentioned
above. The same period also saw the emergence of several African
American \emph{literary} magazines, including \emph{Fire!!} in New York
in 1926, \emph{Black Opals} in Philadelphia in 1927-1928, and
\emph{Saturday Evening Quill} in Boston from 1928-1930. \emph{Fire!!}
ran for only a single issue; \emph{Black Opals} ran for four issues, and
featured many prominent names despite having a fairly tiny print run
(250 copies of each issue were printed). \emph{Saturday Evening Quill},
for its part, was essentially the house magazine for a vibrant African
American literary social club active in the late 1920s in Boston. In the
present collection, we have collected poems from each of these
magazines.

Last but not least, several mainstream publications open their pages to
special issues devoted to African American poetry during this period.
The most prominent of these is undoubtedly the special March 1925 issue
of \emph{Survey Graphic} guest-edited by Alain Locke, which would become
the basis for the hugely influential anthology \emph{The New Negro: an
Interpretation} (1925). There were also special issues dedicated to
African American poetry published by modernist little magazines like
\emph{Palms} (guest-edited by Countee Cullen), as well as \emph{The
Carolina Magazine}, which did excellent issues on African American
poetry, guest-edited by Lewis Alexander. This growing engagement from
the mainstream was also reflected of course in the fact that mainstream
commercial publishers in New York and Boston were also publishing
significant quantities of African American poetry during this time
period. Success in periodical publication often led to book-length
poetry collections -- Langston Hughes, Countee Cullen, and James Weldon
Johnosn all published book-length collections of poetry with mainstream
New York publishers in the subsequent years. (Sadly, writers who were
women did not appear to have this same benefit.)

Alongside the Special Issues described above, Claude McKay published a
number of poems, especially in Max Eastman-edited magazine, \emph{The
Liberator} (not to be confused with William Lloyd Garrison's magazine of
the same name from the mid-19th century). McKay is somewhat anomalous in
that he published much more outside of the Black press than he did
within it (though he is certainly well-represented in \emph{The
Crisis}).

\subsection{\texorpdfstring{\textbf{Where Did the Data Come From? Who
Collected It?
}}{Where Did the Data Come From? Who Collected It? }}\label{where-did-the-data-come-from-who-collected-it}

The texts that form the basis of the data in this dataset come from a
variety of sources, including publicly available resources like
\_HathiTrust \_and \emph{Archive.org}. We also relied on the digitized
versions of \_Saturday Evening Quill \_at the University of Delaware's
library. And a colleague at North Carolina State University, Madhusudan
Kutti, was kind enough to send us a digitized copy of the 1926 special
issue of \emph{Palms} we have included in the collection; it was not
available elsewhere. Our own work entailed reading the issues, locating
poetry published in them, and creating text versions. We also made an
effort to describe the poetic form, mark down thematic tags (``Progress
and Racial Uplift''; ``Motherhood''; etc.). (See the Glossary of Themes
and Forms Tab for more)

The lead researcher on this project is Amardeep Singh, a professor of
English at Lehigh University. He has been creating digital projects
related to minoritized writers for several years, beginning with
projects like ``Claude McKay's Early Poetry'' and ``Women of the Early
Harlem Renaissance.'' Since 2022, he has been developing a larger
project called ``African American Poetry: A Digital Anthology''
(AAPADA), a Scalar project that aims to be a comprehensive collection of
poetry by Black writers that is out of copyright. The poems collected
for this dataset are also presented in simple text versions on AAPADA.

Kate Hennessey, a Lehigh University graduate student, assisted with the
collection of the data as well as the editing and formatting of poems
during the spring of 2024. Other Lehigh graduate students have been
involved with the project, either as paid graduate research assistants
or as students in seminars, whose research and editing skills have at
times been added to the project (with full credit and permission). Other
recent students who have contributed to the project include Miranda
Alvarez Guillen and Christian Farrior. Several of the students involved
in the project are people who identify as Black.

\section{Explore the Data}

\begin{Shaded}
\begin{Highlighting}[]
\NormalTok{//| echo: false}
\NormalTok{//| output: false}
\NormalTok{aa\_poetry\_data = d3.csv("https://raw.githubusercontent.com/melaniewalsh/responsible{-}datasets{-}in{-}context/main/datasets/aa{-}periodical{-}poetry/AAPADA{-}Periodical{-}Poetry\_1900{-}1928.csv", d3.autoType)}
\end{Highlighting}
\end{Shaded}

\begin{Shaded}
\begin{Highlighting}[]
\NormalTok{//| echo: false}
\NormalTok{//| output: false}


\NormalTok{filtered = aa\_poetry\_data.filter(function(penguin) \{}
\NormalTok{  return bill\_length\_min \textless{} penguin.bill\_length\_mm \&\&}
\NormalTok{         islands.includes(penguin.island);}
\NormalTok{\})}
\end{Highlighting}
\end{Shaded}

\begin{Shaded}
\begin{Highlighting}[]
\NormalTok{//| echo: false}
\NormalTok{color = d3}
\NormalTok{  .scaleLinear()}
\NormalTok{  .domain([0, 100, 300])}
\NormalTok{  .range(["\#cafcc2", "\#fce7c2", "\#eb9494"])}
\end{Highlighting}
\end{Shaded}

\subsection{African American Periodical
Poetry}\label{african-american-periodical-poetry}

\begin{Shaded}
\begin{Highlighting}[]
\NormalTok{//| echo: false}
\NormalTok{viewof search = Inputs.search(aa\_poetry\_data, \{}
\NormalTok{  placeholder: "Search"}
\NormalTok{\})}
\end{Highlighting}
\end{Shaded}

\begin{Shaded}
\begin{Highlighting}[]
\NormalTok{//| echo: false}

\NormalTok{/*Inputs.table(search, data)*/}

\NormalTok{Inputs.table(search, \{}
\NormalTok{  layout: "fixed",}
\NormalTok{  rows: 50,}
\NormalTok{  sort: "year",}
\NormalTok{  reverse: false,}
\NormalTok{  format: \{}
\NormalTok{    /*RecreationVisits: x =\textgreater{} d3.format(\textquotesingle{}.2s\textquotesingle{})(x),*/}
\NormalTok{    year: x =\textgreater{} d3.timeFormat(x),}
\NormalTok{    text: x =\textgreater{} \{}
\NormalTok{      const words = x.split(\textquotesingle{} \textquotesingle{});}
\NormalTok{      return words.slice(0, 15).join(\textquotesingle{} \textquotesingle{}) + (words.length \textgreater{} 15 ? \textquotesingle{}...\textquotesingle{} : \textquotesingle{}\textquotesingle{});}
\NormalTok{    \}}
  
\NormalTok{  \}}
\NormalTok{\})}
\end{Highlighting}
\end{Shaded}

\section{How to Use}

\subsection{\texorpdfstring{\textbf{How might this dataset -- and the
collection of texts behind it -- be used?
}}{How might this dataset -- and the collection of texts behind it -- be used? }}\label{how-might-this-dataset-and-the-collection-of-texts-behind-it-be-used}

There are any number of ways this collection might be used.

\textbf{Literary Prestige and Canonization.} While some poets saw their
careers and reputations take off during the peak years of the Harlem
Renaissance, other writers were largely left by the wayside. By
examining the \textbf{``Prize-Winning'' }keyword under ``Themes,'' users
can zoom in on poems and poets who dominated the literary prize contests
at magazines like \_Opportunity \_and \_The Crisis \_starting in the
mid-1920s. Another way of exploring the impact of individual poems might
be to note poems that were republished in multiple venues (see the
column \textbf{``Second Venue''} in the dataset along those lines). The
fact that a poem was repeatedly reprinted might be an indication of its
literary cachet and social impact.

\textbf{Gender.} Another use might be to study changing gender
distributions over the time period at issue. In the early years of the
20th century, poetry published in magazines like \_Colored American
Magazine \_and \emph{Voice of the Negro} was overwhelmingly dominated by
poets who were men. By the 1920s, there was a significant uptick in
poetry by women; how and where this was occurring could be studied in
greater detail.

Another intriguing question is the role of editors who were women.
Jessie Fauset played an important role at \emph{The Crisis} through the
1910s through the mid-1920s; editors who were women also played
important roles at \emph{Colored American Magazine} (Pauline Hopkins),
and \emph{Black Opals} (Nellie Bright, Gwendolyn Bennett).

As of this writing, we are unaware of any African American poets from
this period who identified as transgender or non-binary in our
collection. If we come to learn more about authors' gender identity that
might lead us to add additional gender identity categories, we will do
that.

\textbf{Poetic Form.} Scholars interested in the evolution of African
American poetic forms in the early 20th century might want to explore
our Poetic Form column. At the beginning of this period, most published
poetry by Black poets was in rhymed, accented verse (often using the
\textbf{Common Measure}); following the dominant trend in American
poetry in the mid-1920s, there was an explosion of \textbf{Free Verse}
in many of the magazines. The collection also contains a large number of
African American \textbf{Sonnets} (written throughout the period), as
well as experiments with the \textbf{Blues}, \textbf{Elegies}, and a
number of other forms.

\textbf{The Impact of Editors and Author-Editors.} Editors played a huge
role in encouraging African American poets in these various magazines.
Some editors were also poets themselves. The poet Countee Cullen was on
the masthead as an Associate Editor at \emph{Opportunity} during the
peak of his fame (1927-1928); he also guest-edited a special issue of
the little magazine \emph{Palms} in 1926, which probably informed his
subsequent decision to edit the 1927 collection, \emph{Caroling Dusk}.
Jessie Fauset, Gwendolyn Bennett, and Lewis Alexander were all poets who
also served as guest-editors or associate editors in various magazines
during this period.

\textbf{Predominantly Black vs.~Predominantly White Publications. }We
have marked whether the magazines where poems where published were
``predominantly White'' or ``predominantly Black.'' In the 1920s in
particular, there was a striking increase in the number of predominantly
White publications that ran special issues devoted to African American
poetry. Some of these were especially high-impact events (especially the
special issue of \emph{Survey Graphic} in March 1925). This too could be
studied, especially in connection with other attributes (for instance,
one sees that the \emph{Survey Graphic} special issue mentioned above
was especially influential; but the issue is far more lopsided towards
poets who were men than one sees in predominantly Black publications
from the same time period).

\textbf{White poets? }As magazines like \emph{The Crisis} and
\emph{Opportunity} became more prominent as national publications in the
1920s, they also had a substantial readership that was White. And with
Countee Cullen playing a prominent role as an Associate Editor at
\emph{Opportunity} in 1927-1928, there were quite a few poets who were
White publishing in these magazines. At present, we have excluded these
writers from the collection. A future version of this project might
include White poets who engaged with racial justice issues in the Black
press (there were many who did, and their work has not been given much
critical attention).

\textbf{Frequency of publication and authorial prestige. }Poets like
Langston Hughes and Countee Cullen published many poems in several
different magazines during this period (both predominantly Black and
predominantly White magazines). But poets who are less well-known today
were also highly prolific -- one thinks especially of Georgia Douglas
Johnson (who has approximately 50 poems represented in this dataset).
However, some writers who published much less went on to have an
outsized impact on the basis of relatively few poems -- one thinks of
Helene Johnson, Mae V. Cowdery, and Arna Bontemps along those lines.

\textbf{Exploring lesser-known authors. }Many poems in the dataset were
published by poets who did not aspire to have careers as poets; they may
have only published one or two poems. Others (such as Edward S. Silvera
and Lewis Alexander) were accomplished poets who were well-known to
their peers, but might have dropped off abruptly (Edward Silvera died
quite young, at age 31). And there are a few authors who published five
or more poems included in this collection about whom we have been able
to find nothing (one such name might be Ann Lawrence {[}also known as
Ann Lawrence-Lucas{]}; she published several thoughtful poems in
\emph{The Messenger} in the mid-1920s).

\section{Exercises}

\section{Python}

\phantomsection\label{exercise-posts}

\section{R}

\section{Discussion \& Activities}

\subsubsection{1. Gender distribution over time or by
venue.}\label{gender-distribution-over-time-or-by-venue.}

First, sort the dataset by year. For each decade, check the gender of
the poet. Is there an increase in poets who are women?

By venue. First, sort by Venue. Then compare the gender distribution in
\emph{The Crisis} against other predominantly Black publications like
\emph{The Messenger} or \emph{Opportunity}. Were there magazines that
tended to publish more poets who were women?

Another question to ask might be the influence of Editors who were
women. Magazines like \emph{Black Opals} and \emph{The Crisis} between
1919-1926 had Editors or Associate Editors who were women. Did the
presence of female Editors lead to more women getting published in the
magazine?

Another venue question: how did gender distribution look in the special
issues of predominantly White magazines like \emph{The Carolina
Magazine}, \emph{Palms}, \_or \emph{Survey Graphic}?

\subsubsection{2. Studying Poetic Form.}\label{studying-poetic-form.}

Poetic form has been tagged, somewhat subjectively and admittedly
imperfectly, by the creators of this dataset, often with reference to
Lewis Turco's \emph{Book of Forms}(2020) as a guide.

\textbf{Free verse.} In the dataset, sort by Poetic form. Was there an
increase in the number of poems tagged as Free Verse poems over time?
When did that increase happen, and which magazines really seemed to
embrace and encourage it?

\textbf{Sonnets.} One really striking feature of this collection is the
prevalence of sonnets. Sonnets are often studied as a form associated
with early modern love poetry -- one thinks of Shakespeare's famous
Sonnets along those lines. However, it was a form that was hugely
popular among African American poets as well, including throughout the
early part of the 20th century. Many African American poets adapted the
form to align it with social justice themes.

First, sort the dataset by ``Form,'' and perhaps paste just the poems
tagged as

Sonnets into a new Google Sheet (there should be about 70 such poems).

Now, sort again by ``Theme,'' and pick a Theme that you are especially
interested in. (One we might recommend exploring might be ``Progress and
Racial Uplift'').

Now, read some of the poems themselves. What patterns do you see?

\textbf{No form -- can you find one?} Take a look at some poems that are
currently not tagged according to form. Can you ascertain a form
yourself? (Note: sometimes it can be interesting to paste the entire
text of a poem into ChatGPT or another generative AI platform and ask
for help determining poetic form. The answers are sometimes accurate.)

\emph{If you discover any poems that you believe are incorrectly tagged,
or where we have not put down a tag, we would be grateful if you would
let us know!}

\subsubsection{3. Themes and Topics.}\label{themes-and-topics.}

We have tagged poems by Theme, with a focus on themes relating to social
justice topics. In part, this area of focus relates to patterns that are
observable within the poetry itself. That said, we also expect that
these themes may help readers today see how these poems might be
relevant to our present moment. As with form, this process of tagging by
theme is admittedly somewhat subjective -- it comes out of our own
personal interpretations of the poems. Readers might disagree or think
of very different ways of organizing the poems by theme.

(For example, we have not tagged the many seasonal and occasional poems
in this collection -- poems celebrating Christmas, Easter, Thanksgiving,
etc. -- though readers interested in those topics would likely find them
simply by searching the collection for those keywords. We have also not
tagged poems related to romantic love in the collection -- there are
many!). The general approach we have taken comes from the main
\emph{African American Poetry: A Digital Anthology} website, and a list
of some of the prominent thematic tags (as well as an explanation of how
we're defining them) can be found on the ``Glossary of Themes and
Forms'' Tab.

Sort the collection by ``Theme,'' and explore some of the Themes we have
labeled. Do you agree or disagree with our approach?

Take a look at some poems that are currently un-Tagged. How might you
tag them?

(\emph{Again -- any work researchers / students do here would be of
interest to us, so please send it our way!})

\subsubsection{4. The Crossover
Phenomenon}\label{the-crossover-phenomenon}

As mentioned above, the 1920s was a moment when predominantly White
magazines and publishers opened their doors to African American writers
for the first time -- one might think of it as the seeds of integration
of the publishing industry. Three such magazines that feature
prominently in our dataset are \emph{Survey Graphic}, \emph{Palms}, and
\emph{The Carolina Magazine}.

We have tagged magazines in the dataset by Type of Venue, and the
different venues could be compared and studied.

As a more advanced query, one could layer this question over top of the
thematic question: did the poems published by African American authors
in predominantly White venues look different thematically than poems
published in predominantly Black venues like \emph{The Crisis}. And how
did the gender distribution compare?

\subsubsection{5. Social Network
Analysis.}\label{social-network-analysis.}

This is a more advanced topic, but the seeds of a fairly complex social
network should be visible in this dataset.

Researchers (and students) with skills constructing data visualizations
using software like Gephi or Tableau might wish to extract just author
information and venue information.

A more complex visualization could focus on the relationship between
editors and authors (which gets even more interesting when the authors
become editors).

\subsubsection{5. Researching Unknown
Authors.}\label{researching-unknown-authors.}

As mentioned above, there are a number of authors in the collection
about whom little is known. All told, there are about 100 poems in this
collection, by 70 different authors, whose personal biographies have not
been verified.

We have typically spent \emph{some} time attempting to locate each of
the unknown authors, but there is probably more that could be done,
especially using reference texts like \emph{Encylcopedia of the Harlem
Renaissance} (a print reference text), ``Who's Who?'' type books, and
Census information.

Pick an author about whom we haven't been able to find anything concrete
(the ``Author Bio'' column is left blank for these). Try various ways of
searching, including simple Google searches, searches in HathiTrust,
WorldCat, and VIAF searches. Does anything come up? Can you verify the
identity of the author?

\section{Glossary of Topics}

\subsection{Glossary: Topics, Themes, and
Forms}\label{glossary-topics-themes-and-forms}

As we've been adding poetry to this collection, we've been marking
various thematic areas of interest. These tags can be a good way to
explore our collection and see patterns and begin to emerge between and
among authors featured on this site.

\textbf{Historical Events and Tribute Poems}

\begin{itemize}
\tightlist
\item
  \textbf{World War I}. This tag will be especially interesting to
  readers interested in African American writers responding to World War
  I, especially in the 1910s. Some writers took a critical eye to how
  Black soldiers were treated, and the difficulty of mustering patriotic
  feeling in a society that treated Black folks as second-class
  citizens. \emph{(14 poems with this tag)}
\item
  \textbf{Spanish-American War}. Poems dealing with African American
  soldiers fighting in the Spanish-American war, including the
  Philippine War that followed (mainly 1900-1910).\\
\item
  \textbf{Slavery}. Many of these writers were writing in living memory
  of the slavery era, and had relatives who had been enslaved. Paul
  Laurence Dunbar, for instance, was born to parents who had been
  enslaved. For Black writers in this period, grappling with the legacy
  of slavery was an important and ongoing topic. \emph{(33 poems with
  this tag)}
\item
  \textbf{Lynching and Racialized Violence}. Thousands of African
  American people were brutally murdered in extrajudicial mob killings
  -- lynchings -- in this period. This was a topic that was widely
  covered in the Black press, especially in magazines like \emph{The
  Crisis}. A few poems directly address this horrific phenomenon. Under
  this tag, we also include poems dealing with racialized violence (such
  as race riots) that might not strictly speaking be understood as
  lynchings. \emph{(21 poems with this tag)}
\item
  \textbf{Frederick Douglass}. Tribute poems for Frederick Douglass.
\item
  \textbf{Paul Laurence Dunbar}. Tribute poems for Paul Laurence Dunbar.
\item
  \textbf{Abraham Lincoln}. Poems by Black poets commemorating President
  Abraham Lincoln.
\item
  \textbf{Civil War}. Poems by Black poets reflecting on the Civil War
  (1861-1865).
\end{itemize}

\textbf{Social and Political Themes}

\begin{itemize}
\tightlist
\item
  \textbf{Race}. Many, many poems in this collection might be tagged as
  dealing with race in some way, so over time we have added additional
  differentiations for two especially important subtopics:
  \textbf{``Race: Identity formation''} and \textbf{``Race: Black
  Beauty.''} \emph{(100+ poems with some version of this tag)}
\item
  \textbf{Racism}. Poems tagged with ``racism'' deal explicitly with
  racialized hostility, segregation, and its effects on the African
  American community.
\item
  \textbf{Progress and Racial Uplift}. Poems thematizing racial
  progress, uplift, and the civil rights movement. \emph{(Approximately
  44 poems in the collection are tagged this way)}
\item
  \textbf{Queer and Homoerotic}. Poems dealing with same-sex desire or
  LGBTQIA identities. Several important writers from this period are
  known from biographical evidence to have been LGBTQIA; here we have
  tagged only poems that seem to explicitly reflect same-sex or queer
  desire.
\item
  \textbf{Motherhood}. This is a rich topic, especially for poets from
  the 1910s. Some of the most memorable poetry thematizing motherhood is
  written by Georgia Douglas Johnson. \emph{(18 poems with this tag)}
\item
  \textbf{Interracial, Multiracial, and Race Relations}. Poems dealing
  with relationships that cross racial borders, including friendships,
  antagonistic relationships, and romances. We also use this category
  for poems that reference mixed-race people, such as Georgia Douglas
  Johnson's ``The Octoroon.'' \emph{(21 poems with this tag)}
\item
  \textbf{Religion}. Poems representing engagement with religion,
  including both the affirmative embrace of Christianity in the interest
  of civil rights and more critical engagements.
\item
  \textbf{Labor. }Poems dealing with class and labor relations.
  \emph{(approx. 30 poems with this tag)}
\item
  \textbf{Travel, Migration and Great Migration}. This was a period in
  which many African American writers were on the move, and this tag
  references poems that deal with travel in Europe or Africa. It is also
  used for poems dealing with the Great Migration, the massive movement
  of Black folks from the U.S. south to northern amd midwestern cities
  that began in the early 1900s and continued through the 1940s.
  \emph{(approx. 28 poems with this tag)}
\item
  \textbf{Patriotism}. Poems exploring the often conflicted relationship
  many African American writers had with questions of patriotism and
  national pride.
\end{itemize}

\textbf{Places / Regions / Institutions}

\begin{itemize}
\tightlist
\item
  \textbf{Harlem}. As is well-known, Harlem was the epicenter of the
  Black cultural community that emerged in the 1920s, and many poets
  celebrated it in their writing, including Claude McKay, Langston
  Hughes, and many others.
\item
  \textbf{HBCU}. Many Black writers from this period were associated
  with Historically Black Colleges and Universities in some way. Some --
  notably Langston Hughes -- attended HBCUs and received degrees there.
  Others, including Alain Locke, W.E.B. Du Bois, Charles Johnson, and
  Zora Neale Hurston, taught at HBCUs. There are many references to
  universities like Fisk University, Howard University, Atlanta
  University, and so on in the poetry of this period.
\item
  \textbf{Africa}. Poems that thematize Africa in some way. Imagining
  Africa was an important theme for many Black poets in the early 20th
  century -- both those involved with the UNIA and Garveyism and more
  broadly. Africa was especially vivid to Langston Hughes, as he had a
  memorable visit to West Africa early in his career.\\
\item
  \textbf{Caribbean. }Writers like Claude McKay and Eric Walrond had
  direct Caribbean connections, and often wrote about it in their works.
  But others -- especially Langston Hughes -- also engaged the Caribbean
  in their writing. The Caribbean was also important to Arthur
  Schomburg, who immigrated to the U.S. from Puerto Rico in 1891.
\end{itemize}

\textbf{Other Themes and Forms}

\begin{itemize}
\tightlist
\item
  \textbf{Sonnet}. Poems using the sonnet form. The sonnet form was
  highly prevalent in African American poetry during this period. Claude
  McKay famously used sonnets in many of his most influential political
  poems. Writers like Georgia Douglas Johnson and Carrie Williams
  Clifford also found the sonnet form appealing, again, in connection
  with social justice-oriented poetry. \emph{(approx. 75 poems with this
  tag)}
\item
  Various other poetic forms: \textbf{Common Measure},
  \textbf{Villanelle}, \textbf{Terza Rima}, \textbf{Elegy},
  \textbf{Ode}, \textbf{Haiku}, \textbf{Ballad}, \textbf{Free Verse},
  \textbf{Blues}
\item
  \textbf{Prize-winning poems. }Poems that won one of the poetry prize
  competitions from the mid-1920s, sponsored by either \emph{The Crisis}
  or \emph{Opportunity} magazine. \emph{(32 poems with this tag)}
\item
  \textbf{Music}. The emphasis on music in poems by Langston Hughes is
  well-known, but in fact many Black poets from this period were
  interested in the connections between music and lyricism.
\item
  \textbf{Intertextual}. Poems that allude to other authors, including
  white authors.
\item
  \textbf{Black Vernacular (AAVE). }Poems using AAVE. There were active
  debates at the time among Black poets about the use of language that
  was sometimes seen as caricaturing Black voices.
\item
  \textbf{Poetry for children}. Many well-known African American writers
  of this period wrote poems for children on occasion, including
  especially Langston Hughes and Jessie Fauset.
\end{itemize}



\end{document}
